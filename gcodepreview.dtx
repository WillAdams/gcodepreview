% \iffalse meta-comment
%
% gcodepreview.dtx
% Author: William F. Adams (willadams at aol dot com)
% Copyright 2021--24 William F. Adams
%
% This work may be distributed and/or modified under the
% conditions of the GNU LESSER GENERAL PUBLIC LICENSE
% Version 2.1, February 1999
%
% This work consists of the files listed in the README file.
%
% 
% 
%<*driver>
\documentclass[twoside]{ltxdoc}
\usepackage{docmfp}
\usepackage{url}
\usepackage[draft=false,
            plainpages=false,
            pdfpagelabels,
            bookmarksnumbered,
            hyperindex=false
           ]{hyperref}
\makeatletter
  \@mparswitchfalse
\makeatother
\renewcommand{\MakeUppercase}[1]{#1}
\pagestyle{headings}
\EnableCrossrefs
\CodelineIndex
\usepackage{makeidx} 
%\usepackage[columns=1]{idxlayout}
%%% \OnlyDescription
\setcounter{StandardModuleDepth}{1}
\begin{document}
  \DocInput{gcodepreview.dtx}
\end{document}
%</driver>
%
% \fi
%
% \DoNotIndex{\bullet}
%
%
% \changes{v0.2}{2024/04/12}{Initial conversion to DTX}
% \def\dtxfile{gcodepreview.dtx}
% \def\fileversion{v0.2} \def\filedate{2024/04/12}
%
% \title{The gcodepreview OpenSCAD library\thanks{This
%        file (\texttt{\dtxfile}) has version number \fileversion, last revised
%        \filedate.}}
%
% \author{%
% Author: William F. Adams\\
% \texttt{willadams at aol dot com}
% }
% \date{\filedate}
% \maketitle
% \begin{abstract}
%    The gcodepreview library allows using OpenSCAD to move a tool in lines 
%and output dxf and G-code files so as to work as a CAD/CAM program for CNC.
% \end{abstract}
% \tableofcontents
%
%
% \section{readme.md}
%
%    \begin{macrocode}
%<rdm> # gcodepreview
%<rdm> 
%<rdm> OpenSCAD library for moving a tool in lines and arcs so as to model how a part would be cut 
%<rdm> using G-Code, so as to allow OpenSCAD to function as a compleat CAD/CAM solution for 
%<rdm> subtractive CNC (mills and routers).
%<rdm> 
%<rdm> ![OpenSCAD Cut Joinery Module](https://raw.githubusercontent.com/WillAdams/gcodepreview/main/openscad_cutjoinery.png?raw=true)
%<rdm> 
%<rdm> Updated to make use of Python in OpenSCAD:
%<rdm> 
%<rdm> http://www.guenther-sohler.net/openscad/
%<rdm> 
%<rdm> (previous versions had used RapCAD)
%<rdm> 
%<rdm> A BlockSCAD file for the main modules is available at:
%<rdm> 
%<rdm> https://www.blockscad3d.com/community/projects/1244473
%<rdm> 
%<rdm> The project is discussed at:
%<rdm> 
%<rdm> https://forum.makerforums.info/t/g-code-preview-using-openscad-rapcad/85729 
%<rdm> 
%<rdm> and
%<rdm> 
%<rdm> https://forum.makerforums.info/t/openscad-and-python-looking-to-finally-be-resolved/88171
%<rdm> 
%<rdm> and
%<rdm> 
%<rdm> https://willadams.gitbook.io/design-into-3d/programming
%<rdm> 
%<rdm> Usage is:
%<rdm> 
%<rdm> Place the file in C:\Users\\\~\Documents\OpenSCAD\libraries (C:\Users\\\~\Documents\RapCAD\libraries is 
%<rdm> deprecated since RapCAD is not longer needed since Python is now used for writing out files)
%<rdm> 
%<rdm> (While it was updated for use w/ RapCAD, so as to take advantage of the writeln command, 
%<rdm> it was possible to write that in Python)
%<rdm> 
%<rdm>     use <gcodepreview.py>;
%<rdm>     use <pygcodepreview.scad>;
%<rdm>     include <gcodepreview.scad>;
%<rdm> 
%<rdm> Note that it is necessary to use the first two files (this allows loading 
%<rdm> the Python commands and then wrapping them in OpenSCAD commands) and then 
%<rdm> include the last file (which allows using OpenSCAD variables to selectively 
%<rdm> implement the Python commands via their being wrapped in OpenSCAD modules)
%<rdm> 
%<rdm> and define variables which match the project and then use commands such as:
%<rdm> 
%<rdm>     opengcodefile(Gcode_filename);
%<rdm>     opendxffile(DXF_filename);
%<rdm>     
%<rdm>     difference() {
%<rdm>         setupstock(stocklength, stockwidth, stockthickness, zeroheight, stockorigin);
%<rdm>     
%<rdm>     movetosafez();
%<rdm>     
%<rdm>     toolchange(squaretoolno,speed * square_ratio);
%<rdm>     
%<rdm>     begintoolpath(0,0,0.25);
%<rdm>     beginpolyline(0,0,0.25);
%<rdm> 
%<rdm>     cutoneaxis_setfeed("Z",-1,plunge*square_ratio);
%<rdm>     addpolyline(stocklength/2,stockwidth/2,-stockthickness);
%<rdm>     
%<rdm>     cutwithfeed(stocklength/2,stockwidth/2,-stockthickness,feed);
%<rdm>     
%<rdm>     endtoolpath();
%<rdm>     endpolyline();
%<rdm>     
%<rdm>     }
%<rdm>     
%<rdm>     closegcodefile();
%<rdm>     closedxffile();
%<rdm> 
%<rdm> Tool numbers match those of tooling sold by Carbide 3D (ob. discl., I work for them). 
%<rdm> Comments are included in the G-code to match those expected by CutViewer.
%<rdm> 
%<rdm> A complete example file is: gcodepreview_template.scad another is 
%<rdm> openscad_gcodepreview_cutjoinery.tres.scad which is made from an 
%<rdm> OpenSCAD Graph Editor file:
%<rdm> 
%<rdm> ![OpenSCAD Graph Editor Cut Joinery File](https://raw.githubusercontent.com/WillAdams/gcodepreview/main/OSGE_cutjoinery.png?raw=true)
%<rdm> 
%<rdm> Version 0.1 supports setting up stock, origin, rapid positioning, making cuts, 
%<rdm> and writing out matching G-code, and creating a DXF with polylines.
%<rdm> 
%<rdm> Added features since initial upload:
%<rdm> 
%<rdm>  - endpolyline(); --- this command allows ending one polyline so as to allow multiple lines in a DXF
%<rdm>  - separate dxf files are written out for each tool where tool is ball/square/V and small/large (10/31/23)
%<rdm>  - re-writing as a Literate Program using the LaTeX package docmfp (begun 4/12/24) 
%<rdm> 
%<rdm> Not quite working feature:
%<rdm> 
%<rdm>  - exporting SVGs --- these are written out upside down due to coordinate differences between OpenSCAD/DXFs and SVGs
%<rdm> 
%<rdm> Possible future improvements:
%<rdm> 
%<rdm>  - G-code: support for G2/G3 arcs
%<rdm>  - DXF support for curves and the 3rd dimension
%<rdm>  - G-code: import external tool libraries and feeds and speeds from JSON or CSV files --- note that it is up to the user to implement Depth per Pass so as to not take a single full-depth pass
%<rdm>  - support for additional tooling shapes such as dovetail tools, or roundover tooling
%<rdm>  - general coding improvements --- current coding style is quite prosaic
%<rdm>  - generalized modules for cutting out various shapes/geometries --- a current one is to cut a rectangular area as vertical passes (the horizontal version will be developed presently)
%
%    \end{macrocode}
%
% \section{gcodepreview}
%
% As noted above, this library works by using Python code as a back-end so as to persistently
% store and access variables, and to write out files. Doing so requires a total of three files:
%
% \begin{itemize}
%  \item A Python file: gcodepreview.py (gcpy)
%  \item An OpenSCAD file: gcodepreview.scad (gcpscad)
%  \item An OpenSCAD file which connects the other two files: pygcodepreview.scad (pyscad)
% \end{itemize}
%
% Each file will begin with a suitable comment indicating the file type:
%
%    \begin{macrocode}
%<gcpy>#!/usr/bin/env python
%<gcpy>
%    \end{macrocode}
%
%    \begin{macrocode}
%<pyscad> //!OpenSCAD
%<pyscad> 
%    \end{macrocode}
%
%    \begin{macrocode}
%<gcpscad> //!OpenSCAD
%<gcpscad> 
%<gcpscad> //gcodepreview 0.1
%<gcpscad> //
%<gcpscad> //used via use <gcodepreview.py>;
%<gcpscad> //         use <pygcodepreview.scad>;
%<gcpscad> //         include <gcodepreview.scad>;
%<gcpscad> //
%    \end{macrocode}
%
% The original implementation in RapSCAD used a command \texttt{writeln}\DescribeRoutine{writeln} --- fortunately,
% this command is easily re-created in Python:
%
%    \begin{macrocode}
%<gcpy>def writeln(*arguments):
%<gcpy>    line_to_write = ""
%<gcpy>    for element in arguments:
%<gcpy>        line_to_write += element
%<gcpy>    f.write(line_to_write)
%<gcpy>    f.write("\n")
%    \end{macrocode}
%
% \noindent which command will accept a series of arguments and then write them out to a file object.
%
% \subsection{Position and Variables}
% 
% In modeling the machine motion and G-code it will be necessary to have the machine track 
% several variables. This will be done using paired functions (which will return the matching variable)
% and a matching (global) variable, as well as additional functions for setting the matching variable.
%
% The first such variables are for XYZ position:
%
% \begin{itemize}
%  \item \texttt{mpx} \DescribeVariable{mpx}
%  \item \texttt{mpy} \DescribeVariable{mpy}
%  \item \texttt{mpz} \DescribeVariable{mpz}
% \end{itemize}
%
% It will further be necessary to have a variable for the current tool:
%
% \begin{itemize}
%  \item \texttt{currenttool} \DescribeVariable{currenttool}
% \end{itemize}
%
% For each command it will be necessary to implement an appropriate aspect in each file. 
% The Python file wil manage the Python variables and handle things which can only be done in Python,
% while there will be two OpenSCAD files as noted above, one which calls the Python code (thjs 
% will be \texttt{use}d), while the other will be \texttt{include}d and will be able to access
% and use OpenSCAD variables, as well as implement Customizer options.
% 
% The first such routine will be appropriately enough, to set up the stock, and perform other
% initializations. \DescribeRoutine{psetupstock}
% 
%    \begin{macrocode}
%<gcpy>def psetupstock(stocklength, stockwidth, stockthickness, zeroheight, stockorigin):
%<gcpy>    global mpx
%<gcpy>    mpx = float(0)
%<gcpy>    global mpy
%<gcpy>    mpy = float(0)
%<gcpy>    global mpz
%<gcpy>    mpz = float(0)
%<gcpy>    global currenttool
%<gcpy>    currenttool = 102
%<gcpy>
%    \end{macrocode}
%
% \DescribeRoutine{osetupstock}
%    \begin{macrocode}
%<pyscad> module osetupstock(stocklength, stockwidth, stockthickness, zeroheight, stockorigin) {
%<pyscad>     psetupstock(stocklength, stockwidth, stockthickness, zeroheight, stockorigin);
%<pyscad> }
%<pyscad> 
%    \end{macrocode}
%
% \DescribeRoutine{setupstock}
%    \begin{macrocode}
%<gcpscad> module setupstock(stocklength, stockwidth, stockthickness, zeroheight, stockorigin) {
%<gcpscad>   osetupstock(stocklength, stockwidth, stockthickness, zeroheight, stockorigin);
%<gcpscad> //initialize default tool and XYZ origin
%<gcpscad>   osettool(102);
%<gcpscad>   oset(0,0,0);
%<gcpscad>   if (zeroheight == "Top") {
%<gcpscad>     if (stockorigin == "Lower-Left") {
%<gcpscad>     translate([0, 0, (-stockthickness)]){
%<gcpscad>     cube([stocklength, stockwidth, stockthickness], center=false);
%<gcpscad> if (generategcode == true) {
%<gcpscad> //	owriteone("(setupstock)");
%<gcpscad> owritethree("(stockMin:0.00mm, 0.00mm, -",str(stockthickness),"mm)");
%<gcpscad> owritefive("(stockMax:",str(stocklength),"mm, ",str(stockwidth),"mm, 0.00mm)");
%<gcpscad>     owritenine("(STOCK/BLOCK, ",str(stocklength),", ",str(stockwidth),", ",str(stockthickness),", 0.00, 0.00, ",str(stockthickness),")");
%<gcpscad> }
%<gcpscad> }
%<gcpscad> }
%<gcpscad>      else if (stockorigin == "Center-Left") {
%<gcpscad>     translate([0, (-stockwidth / 2), -stockthickness]){
%<gcpscad>       cube([stocklength, stockwidth, stockthickness], center=false);
%<gcpscad>     if (generategcode == true) {
%<gcpscad> //	owriteone("(setupstock)");
%<gcpscad> owritefive("(stockMin:0.00mm, -",str(stockwidth/2),"mm, -",str(stockthickness),"mm)");
%<gcpscad> owritefive("(stockMax:",str(stocklength),"mm, ",str(stockwidth/2),"mm, 0.00mm)");
%<gcpscad>     owriteeleven("(STOCK/BLOCK, ",str(stocklength),", ",str(stockwidth),", ",str(stockthickness),", 0.00, ",str(stockwidth/2),", ",str(stockthickness),")");
%<gcpscad>     }
%<gcpscad>     }
%<gcpscad>     } else if (stockorigin == "Top-Left") {
%<gcpscad>     translate([0, (-stockwidth), -stockthickness]){
%<gcpscad>       cube([stocklength, stockwidth, stockthickness], center=false);
%<gcpscad> if (generategcode == true) {
%<gcpscad> //	owriteone("(setupstock)");
%<gcpscad> owritefive("(stockMin:0.00mm, -",str(stockwidth),"mm, -",str(stockthickness),"mm)");
%<gcpscad> owritethree("(stockMax:",str(stocklength),"mm, 0.00mm, 0.00mm)");
%<gcpscad> owriteeleven("(STOCK/BLOCK, ",str(stocklength),", ",str(stockwidth),", ",str(stockthickness),", 0.00, ",str(stockwidth),", ",str(stockthickness),")");
%<gcpscad>     }
%<gcpscad>     }
%<gcpscad>     }
%<gcpscad> 	else if (stockorigin == "Center") {
%<gcpscad> //owritecomment("Center");
%<gcpscad>     translate([(-stocklength / 2), (-stockwidth / 2), -stockthickness]){
%<gcpscad>       cube([stocklength, stockwidth, stockthickness], center=false);
%<gcpscad> if (generategcode == true) {
%<gcpscad> //	owriteone("(setupstock)");
%<gcpscad> owriteseven("(stockMin: -",str(stocklength/2),", -",str(stockwidth/2),"mm, -",str(stockthickness),"mm)");
%<gcpscad> owritefive("(stockMax:",str(stocklength/2),"mm, ",str(stockwidth/2),"mm, 0.00mm)");
%<gcpscad> owritethirteen("(STOCK/BLOCK, ",str(stocklength),", ",str(stockwidth),", ",str(stockthickness),", ",str(stocklength/2),", ", str(stockwidth/2),", ",str(stockthickness),")");
%<gcpscad> }
%<gcpscad> }
%<gcpscad> }
%<gcpscad> } else if (zeroheight == "Bottom") {
%<gcpscad> //owritecomment("Bottom");
%<gcpscad>     if (stockorigin == "Lower-Left") {
%<gcpscad>     cube([stocklength, stockwidth, stockthickness], center=false);
%<gcpscad> if (generategcode == true) {
%<gcpscad> //	owriteone("(setupstock)");
%<gcpscad> owriteone("(stockMin:0.00mm, 0.00mm, 0.00mm)");
%<gcpscad> owriteseven("(stockMax:",str(stocklength),"mm, ",str(stockwidth),"mm, ",str(stockthickness),"mm)");
%<gcpscad> owriteseven("(STOCK/BLOCK, ",str(stocklength),", ",str(stockwidth),", ",str(stockthickness),",0.00, 0.00, 0.00)");
%<gcpscad>     }
%<gcpscad> }	else if (stockorigin == "Center-Left") {
%<gcpscad>     translate([0, (-stockwidth / 2), 0]){
%<gcpscad>       cube([stocklength, stockwidth, stockthickness], center=false);
%<gcpscad> if (generategcode == true) {
%<gcpscad> //	owriteone("(setupstock)");
%<gcpscad> owritethree("(stockMin:0.00mm, -",str(stockwidth/2),"mm, 0.00mm)");
%<gcpscad> owriteseven("(stockMax:",str(stocklength),"mm, ",str(stockwidth/2),"mm, ",str(stockthickness),"mm)");
%<gcpscad> owritenine("(STOCK/BLOCK, ",str(stocklength),", ",str(stockwidth),", ",str(stockthickness),",0.00, ",str(stockwidth/2),", 0.00)");
%<gcpscad>     }
%<gcpscad>     } 
%<gcpscad> 	} else if (stockorigin == "Top-Left") {
%<gcpscad>     translate([0, (-stockwidth), 0]){
%<gcpscad>       cube([stocklength, stockwidth, stockthickness], center=false);
%<gcpscad>     }
%<gcpscad> if (generategcode == true) {
%<gcpscad> //	owriteone("(setupstock)");
%<gcpscad> owritethree("(stockMin:0.00mm, -",str(stockwidth),"mm, 0.00mm)");
%<gcpscad> owritefive("(stockMax:",str(stocklength),"mm, 0.00mm, ",str(stockthickness),"mm)");
%<gcpscad> owritenine("(STOCK/BLOCK, ",str(stocklength),", ",str(stockwidth),", ",str(stockthickness),", 0.00, ", str(stockwidth),", 0.00)");
%<gcpscad> }
%<gcpscad> }	else if (stockorigin == "Center") {
%<gcpscad>     translate([(-stocklength / 2), (-stockwidth / 2), 0]){
%<gcpscad>       cube([stocklength, stockwidth, stockthickness], center=false);
%<gcpscad>     }
%<gcpscad> if (generategcode == true) {
%<gcpscad> //	owriteone("(setupstock)");
%<gcpscad> owritefive("(stockMin:-",str(stocklength/2),", -",str(stockwidth/2),"mm, 0.00mm)");
%<gcpscad> owriteseven("(stockMax:",str(stocklength/2),"mm, ",str(stockwidth/2),"mm, ",str(stockthickness),"mm)");
%<gcpscad> owriteeleven("(STOCK/BLOCK, ",str(stocklength),", ",str(stockwidth),", ",str(stockthickness),", ",str(stocklength/2),", ", str(stockwidth/2),", 0.00)");
%<gcpscad> }
%<gcpscad> }
%<gcpscad> }
%<gcpscad> if (generategcode == true) {
%<gcpscad> 	owriteone("G90");
%<gcpscad> 	owriteone("G21");
%<gcpscad> //	owriteone("(Move to safe Z to avoid workholding)");
%<gcpscad> //	owriteone("G53G0Z-5.000");
%<gcpscad> }
%<gcpscad> //owritecomment("ENDSETUP");
%<gcpscad> }
%<gcpscad> 
%    \end{macrocode}
%
% \DescribeRoutine{xpos}\DescribeRoutine{ypos}\DescribeRoutine{zpos}
% It will be necessary to have Python functions which return the current values of the 
% machine position in Cartesian coordinates: 
%
%    \begin{macrocode}
%<gcpy>def xpos():
%<gcpy>    global mpx
%<gcpy>    return mpx
%<gcpy>
%<gcpy>def ypos():
%<gcpy>    global mpy
%<gcpy>    return mpy
%<gcpy>
%<gcpy>def zpos():
%<gcpy>    global mpz
%<gcpy>    return mpz
%<gcpy>
%    \end{macrocode}
%
% \noindent and in turn, functions which set the positions: \DescribeRoutine{psetxpos}
%  \DescribeRoutine{psetypos}
%  \DescribeRoutine{psetzpos}
%
%    \begin{macrocode}
%<gcpy>def psetxpos(newxpos):
%<gcpy>    global mpx
%<gcpy>    mpx = newxpos
%<gcpy>
%<gcpy>def psetypos(newypos):
%<gcpy>    global mpy
%<gcpy>    mpy = newypos
%<gcpy>
%<gcpy>def psetzpos(newzpos):
%<gcpy>    global mpz
%<gcpy>    mpz = newzpos
%<gcpy> 
%    \end{macrocode}
% 
% \noindent and as noted above, there will need to be matching OpenSCAD versions: \DescribeRoutine{getxpos}
% \DescribeRoutine{getypos}
% \DescribeRoutine{getzpos}
% \DescribeRoutine{setxpos}
% \DescribeRoutine{setypos}
% \DescribeRoutine{setzpos}
% 
%    \begin{macrocode}
%<pyscad> function getxpos() = xpos();
%<pyscad> function getypos() = ypos();
%<pyscad> function getzpos() = zpos();
%<pyscad> 
%<pyscad> module setxpos(newxpos) {
%<pyscad> psetxpos(newxpos);
%<pyscad> }
%<pyscad> 
%<pyscad> module setypos(newypos) {
%<pyscad> psetypos(newypos);
%<pyscad> }
%<pyscad> 
%<pyscad> module setzpos(newzpos) {
%<pyscad> psetzpos(newzpos);
%<pyscad> }
%<pyscad> 
%    \end{macrocode}
% 
% \DescribeRoutine{oset}
%    \begin{macrocode}
%<gcpscad> module oset(ex, ey, ez) {
%<gcpscad> setxpos(ex);
%<gcpscad> setypos(ey);
%<gcpscad> setzpos(ez);
%<gcpscad> }
%<gcpscad> 
%    \end{macrocode}
% 
% \subsection{Tools and Changes}
% 
% Similarly Python functions and variables will be used to track and
% set and return the current tool: \DescribeRoutine{psettool}
% \DescribeRoutine{pcurrenttool}
% 
%    \begin{macrocode}
%<gcpy>def psettool(tn):
%<gcpy>    global currenttool
%<gcpy>    currenttool = tn
%<gcpy>
%<gcpy>def pcurrent_tool():
%<gcpy>    global currenttool
%<gcpy>    return currenttool
%<gcpy>
%    \end{macrocode}
% 
% \noindent and matching OpenSCAD modules set and return the current tool: \DescribeRoutine{osettool}
% \DescribeRoutine{currenttool}
% 
%    \begin{macrocode}
%<pyscad> module osettool(tn){
%<pyscad> psettool(tn);}
%<pyscad> 
%<pyscad> function current_tool() = pcurrent_tool();
%<pyscad> 
%    \end{macrocode}
% 
% \noindent and apply the appropriate commands for a \texttt{toolchange}.\DescribeRoutine{toolchange}
% 
%    \begin{macrocode}
%<gcpscad> module toolchange(tool_number,speed) {
%<gcpscad>    osettool(tool_number); 
%<gcpscad> if (generategcode == true) {
%<gcpscad> 	writecomment("Toolpath");
%<gcpscad> 	owriteone("M05");
%<gcpscad> //	writecomment("Move to safe Z to avoid workholding");
%<gcpscad> //	owriteone("G53G0Z-5.000");
%<gcpscad> //    writecomment("Begin toolpath");
%<gcpscad>   if (tool_number == 201) {
%<gcpscad> 	writecomment("TOOL/MILL,6.35, 0.00, 0.00, 0.00");
%<gcpscad>   } else if (tool_number == 202) {
%<gcpscad> 	writecomment("TOOL/MILL,6.35, 3.17, 0.00, 0.00");
%<gcpscad>   } else if (tool_number == 102) {
%<gcpscad> 	writecomment("TOOL/MILL,3.17, 0.00, 0.00, 0.00");
%<gcpscad>   } else if (tool_number == 101) {
%<gcpscad> 	writecomment("TOOL/MILL,3.17, 1.58, 0.00, 0.00");
%<gcpscad>   } else if (tool_number == 301) {
%<gcpscad> 	writecomment("TOOL/MILL,0.03, 0.00, 6.35, 45.00");
%<gcpscad>   } else if (tool_number == 302) {
%<gcpscad> 	writecommment("TOOL/MILL,0.03, 0.00, 10.998, 30.00");
%<gcpscad>   } else if (tool_number == 390) {
%<gcpscad> 	writecomment("TOOL/MILL,0.03, 0.00, 1.5875, 45.00");
%<gcpscad>   }
%<gcpscad>     select_tool(tool_number);
%<gcpscad> 	owritetwo("M6T",str(tool_number));
%<gcpscad> 	owritetwo("M03S",str(speed));
%<gcpscad> }
%<gcpscad> }
%<gcpscad> 
%    \end{macrocode}
% 
% There must also be a module for selecting tools: \texttt{select\_tool}:\DescribeRoutine{selecttool}
% \DescribeVariable{tool number}
% 
%    \begin{macrocode}
%<gcpscad> module select_tool(tool_number) {
%<gcpscad> //echo(tool_number);
%<gcpscad>   if (tool_number == 201) {
%<gcpscad>     gcp_endmill_square(6.35, 19.05);
%<gcpscad>   } else if (tool_number == 202) {
%<gcpscad>     gcp_endmill_ball(6.35, 19.05);
%<gcpscad>   } else if (tool_number == 102) {
%<gcpscad>     gcp_endmill_square(3.175, 19.05);
%<gcpscad>   } else if (tool_number == 101) {
%<gcpscad>     gcp_endmill_ball(3.175, 19.05);
%<gcpscad>   } else if (tool_number == 301) {
%<gcpscad>     gcp_endmill_v(90, 12.7);
%<gcpscad>   } else if (tool_number == 302) {
%<gcpscad>     gcp_endmill_v(60, 12.7);
%<gcpscad>   } else if (tool_number == 390) {
%<gcpscad>     gcp_endmill_v(90, 3.175);
%<gcpscad>   }
%<gcpscad> }
%<gcpscad> 
%    \end{macrocode}
% 
% Each tool must be modeled in 3D using an OpenSCAD module:
% 
% \DescribeRoutine{gcp endmill square}
%    \begin{macrocode}
%<gcpscad> module gcp_endmill_square(es_diameter, es_flute_length) {
%<gcpscad>   cylinder(r1=(es_diameter / 2), r2=(es_diameter / 2), h=es_flute_length, center=false);
%<gcpscad> }
%<gcpscad> 
%    \end{macrocode}
% 
% \DescribeRoutine{gcp endmill ball}
%    \begin{macrocode}
%<gcpscad> module gcp_endmill_ball(es_diameter, es_flute_length) {
%<gcpscad>   translate([0, 0, (es_diameter / 2)]){
%<gcpscad>     union(){
%<gcpscad>       sphere(r=(es_diameter / 2));
%<gcpscad>       cylinder(r1=(es_diameter / 2), r2=(es_diameter / 2), h=es_flute_length, center=false);
%<gcpscad>     }
%<gcpscad>   }
%<gcpscad> }
%<gcpscad> 
%    \end{macrocode}
% 
% \DescribeRoutine{gcp endmill v}
%    \begin{macrocode}
%<gcpscad> module gcp_endmill_v(es_v_angle, es_diameter) {
%<gcpscad>   union(){
%<gcpscad>     cylinder(r1=0, r2=(es_diameter / 2), h=((es_diameter / 2) / tan((es_v_angle / 2))), center=false);
%<gcpscad>     translate([0, 0, ((es_diameter / 2) / tan((es_v_angle / 2)))]){
%<gcpscad>       cylinder(r1=(es_diameter / 2), r2=(es_diameter / 2), h=((es_diameter * 8) ), center=false);/// tan((es_v_angle / 2))
%<gcpscad>     }
%<gcpscad>   }
%<gcpscad> }
%<gcpscad> 
%    \end{macrocode}
% 
% \subsection{File Handling}
% 
% For writing to files it will be necessary to have commands for each step of working with the files. 
% 
% \DescribeRoutine{popengcodefile}\DescribeRoutine{popendxffile}\DescribeRoutine{popendxlgblffile}
% \DescribeRoutine{popendxflgsqfile}\DescribeRoutine{popendxflgVfile}\DescribeRoutine{popendxfsmblfile}
% There is a separate function for each type of file, and for DXFs, there are multiple file instances, 
% one for each type of different type and size of tool which it is expected a project will work with.
% \DescribeRoutine{popendxfsmsqfile}\DescribeRoutine{popendxfsmVfile}\DescribeRoutine{popensvgfile}
% 
%    \begin{macrocode}
%<gcpy>def popengcodefile(fn):
%<gcpy>    global f
%<gcpy>    f = open(fn, "w")
%<gcpy>
%<gcpy>def popendxffile(fn):
%<gcpy>    global dxf
%<gcpy>    dxf = open(fn, "w")
%<gcpy>
%<gcpy>def popendxlgblffile(fn):
%<gcpy>    global dxflgbl
%<gcpy>    dxflgbl = open(fn, "w")
%<gcpy>
%<gcpy>def popendxflgsqfile(fn):
%<gcpy>    global dxfldsq
%<gcpy>    dxflgsq = open(fn, "w")
%<gcpy>
%<gcpy>def popendxflgVfile(fn):
%<gcpy>    global dxflgV
%<gcpy>    dxflgV = open(fn, "w")
%<gcpy>
%<gcpy>def popendxfsmblfile(fn):
%<gcpy>    global dxfsmbl
%<gcpy>    dxfsmbl = open(fn, "w")
%<gcpy>
%<gcpy>def popendxfsmsqfile(fn):
%<gcpy>    global dxfsmsq
%<gcpy>    dxfsmsq = open(fn, "w")
%<gcpy>
%<gcpy>def popendxfsmVfile(fn):
%<gcpy>    global dxfsmV
%<gcpy>    dxfsmV = open(fn, "w")
%<gcpy>
%<gcpy>def popensvgfile(fn):
%<gcpy>    global svg
%<gcpy>    svg = open(fn, "w")
%<gcpy>
%    \end{macrocode}
% 
% \DescribeRoutine{oopengcodefile}\DescribeRoutine{oopensvgfile}\DescribeRoutine{oopendxffile}
% There will need to be matching OpenSCAD modules for the Python functions.
%    \begin{macrocode}
%<pyscad> module oopengcodefile(fn) {
%<pyscad> 	popengcodefile(fn);
%<pyscad> }
%<pyscad> 
%<pyscad> module oopensvgfile(fn) {
%<pyscad> 	popensvgfile(fn);
%<pyscad> }
%<pyscad> 
%<pyscad> module oopendxffile(fn) {
%<pyscad>     echo(fn);
%<pyscad> 	popendxffile(fn);
%<pyscad> }
%<pyscad> 
%<pyscad> module oopendxflgblfile(fn) {
%<pyscad> 	popendxflgblfile(fn);
%<pyscad> }
%<pyscad> 
%<pyscad> module oopendxflgsqfile(fn) {
%<pyscad> 	popendxflgsqfile(fn);
%<pyscad> }
%<pyscad> 
%<pyscad> module oopendxflgVfile(fn) {
%<pyscad> 	popendxflgVfile(fn);
%<pyscad> }
%<pyscad> 
%<pyscad> module oopendxfsmblfile(fn) {
%<pyscad> 	popendxfsmblfile(fn);
%<pyscad> }
%<pyscad> 
%<pyscad> module oopendxfsmsqfile(fn) {
%<pyscad>     echo(fn);
%<pyscad> 	popendxfsmsqfile(fn);
%<pyscad> }
%<pyscad> 
%<pyscad> module oopendxfsmVfile(fn) {
%<pyscad> 	popendxfsmVfile(fn);
%<pyscad> }
%<pyscad> 
%    \end{macrocode}
% 
% \DescribeRoutine{opengcodefile}\DescribeRoutine{opensvgfile}\DescribeRoutine{opendxffile}
%    \begin{macrocode}
%<gcpscad> module opengcodefile(fn) {
%<gcpscad> if (generategcode == true) {
%<gcpscad> 	oopengcodefile(fn);
%<gcpscad>     echo(fn);
%<gcpscad>     owritecomment(fn);
%<gcpscad> }
%<gcpscad> }
%<gcpscad> 
%<gcpscad> module opensvgfile(fn) {
%<gcpscad> if (generatesvg == true) {
%<gcpscad> 	oopensvgfile(fn);
%<gcpscad>     echo(fn);
%<gcpscad>     svgwriteone(str("<?xml version=",chr(34),"1.0",chr(34)," encoding=",chr(34),"UTF-8",chr(34)," standalone=",chr(34),"no",chr(34),"?> "));
%<gcpscad> //	writesvglineend();
%<gcpscad> svgwriteone(str("<svg  version=",chr(34),"1.1",chr(34)," xmlns=",chr(34),"http://www.w3.org/2000/svg",chr(34)," width=",chr(34),stocklength*3.77953,"px",chr(34)," height=",chr(34),stockwidth*3.77953,"px",chr(34),"> "));
%<gcpscad> //<path d="M755.906 0 L755.906 377.953 L0 377.953 L0 0 L755.906 0 Z " stroke="black" stroke-width="1" fill="none" /> 
%<gcpscad> svgwriteone(str("<path d=",chr(34),"M",stocklength*3.77953," 0 L",stocklength*3.77953," ",stockwidth*3.77953," L0 ",stockwidth*3.77953," L0 0 L",stocklength*3.77953," 0 Z ",chr(34)," stroke=",chr(34),"black",chr(34)," stroke-width=",chr(34),"1",chr(34)," fill=",chr(34),"none",chr(34)," /> "));
%<gcpscad>     }
%<gcpscad> }
%<gcpscad> 
%<gcpscad> module opendxffile(fn) {
%<gcpscad> if (generatedxf == true) {
%<gcpscad> 	oopendxffile(str(fn,".dxf"));
%<gcpscad> //    echo(fn);
%<gcpscad>     dxfwriteone("0");
%<gcpscad>     dxfwriteone("SECTION");
%<gcpscad>     dxfwriteone("2");
%<gcpscad>     dxfwriteone("ENTITIES");
%<gcpscad>     dxfwriteone("0");
%<gcpscad> if (large_ball_tool_no >  0) {	oopendxflgblfile(str(fn,".",large_ball_tool_no,".dxf"));
%<gcpscad>     dxfpreamble(large_ball_tool_no);
%<gcpscad> } 
%<gcpscad> if (large_square_tool_no >  0) {	oopendxflgsqfile(str(fn,".",large_square_tool_no,".dxf"));
%<gcpscad>     dxfpreamble(large_square_tool_no);
%<gcpscad> } 
%<gcpscad> if (large_V_tool_no >  0) {	oopendxflgVfile(str(fn,".",large_V_tool_no,".dxf"));
%<gcpscad>     dxfpreamble(large_V_tool_no);
%<gcpscad> } 
%<gcpscad> if (small_ball_tool_no >  0) {	oopendxfsmblfile(str(fn,".",small_ball_tool_no,".dxf"));
%<gcpscad>     dxfpreamble(small_ball_tool_no);
%<gcpscad> } 
%<gcpscad> if (small_square_tool_no >  0) {	oopendxfsmsqfile(str(fn,".",small_square_tool_no,".dxf"));
%<gcpscad> //    echo(str("tool no",small_square_tool_no));
%<gcpscad>     dxfpreamble(small_square_tool_no);
%<gcpscad> } 
%<gcpscad> if (small_V_tool_no >  0) {	oopendxfsmVfile(str(fn,".",small_V_tool_no,".dxf"));
%<gcpscad>     dxfpreamble(small_V_tool_no);
%<gcpscad> } 
%<gcpscad> }
%<gcpscad> }
%<gcpscad> 
%    \end{macrocode}
% 
% \DescribeRoutine{writedxf}\DescribeRoutine{writedxflgbl}\DescribeRoutine{writedxflgsq}
% Once files have been opened they may be written to.
% 
%    \begin{macrocode}
%<gcpy>def writedxf(*arguments):
%<gcpy>    line_to_write = ""
%<gcpy>    for element in arguments:
%<gcpy>        line_to_write += element
%<gcpy>    dxf.write(line_to_write)
%<gcpy>    dxf.write("\n")
%<gcpy>
%<gcpy>def writedxflgbl(*arguments):
%<gcpy>    line_to_write = ""
%<gcpy>    for element in arguments:
%<gcpy>        line_to_write += element
%<gcpy>    dxflgbl.write(line_to_write)
%<gcpy>    print(line_to_write)
%<gcpy>    dxflgbl.write("\n")
%<gcpy>
%<gcpy>def writedxflgsq(*arguments):
%<gcpy>    line_to_write = ""
%<gcpy>    for element in arguments:
%<gcpy>        line_to_write += element
%<gcpy>    dxflgsq.write(line_to_write)
%<gcpy>    print(line_to_write)
%<gcpy>    dxflgsq.write("\n")
%<gcpy>
%<gcpy>def writedxflgV(*arguments):
%<gcpy>    line_to_write = ""
%<gcpy>    for element in arguments:
%<gcpy>        line_to_write += element
%<gcpy>    dxflgV.write(line_to_write)
%<gcpy>    print(line_to_write)
%<gcpy>    dxflgV.write("\n")
%<gcpy>
%<gcpy>def writedxfsmbl(*arguments):
%<gcpy>    line_to_write = ""
%<gcpy>    for element in arguments:
%<gcpy>        line_to_write += element
%<gcpy>    dxfsmbl.write(line_to_write)
%<gcpy>    print(line_to_write)
%<gcpy>    dxfsmbl.write("\n")
%<gcpy>
%<gcpy>def writedxfsmsq(*arguments):
%<gcpy>    line_to_write = ""
%<gcpy>    for element in arguments:
%<gcpy>        line_to_write += element
%<gcpy>    dxfsmsq.write(line_to_write)
%<gcpy>    print(line_to_write)
%<gcpy>    dxfsmsq.write("\n")
%<gcpy>
%<gcpy>def writedxfsmV(*arguments):
%<gcpy>    line_to_write = ""
%<gcpy>    for element in arguments:
%<gcpy>        line_to_write += element
%<gcpy>    dxfsmV.write(line_to_write)
%<gcpy>    print(line_to_write)
%<gcpy>    dxfsmV.write("\n")
%<gcpy>
%<gcpy>def writesvg(*arguments):
%<gcpy>    line_to_write = ""
%<gcpy>    for element in arguments:
%<gcpy>        line_to_write += element
%<gcpy>    svg.write(line_to_write)
%<gcpy>    print(line_to_write)
%<gcpy>
%<gcpy>def pwritesvgline():
%<gcpy>    svg.write("\n")
%<gcpy>
%    \end{macrocode}
% 
% \DescribeRoutine{owritecomment}\DescribeRoutine{dxfwriteone}\DescribeRoutine{dxfwritelgbl}
%    \begin{macrocode}
%<pyscad> module owritecomment(comment) {
%<pyscad> 	writeln("(",comment,")");
%<pyscad> }
%<pyscad> 
%<pyscad> module dxfwriteone(first) {
%<pyscad> 	writedxf(first);
%<pyscad> //	writeln(first);
%<pyscad> //    echo(first);
%<pyscad> }
%<pyscad> 
%<pyscad> module dxfwritelgbl(first) {
%<pyscad> 	writedxflgbl(first);
%<pyscad> }
%<pyscad> 
%<pyscad> module dxfwritelgsq(first) {
%<pyscad> 	writedxflgsq(first);
%<pyscad> }
%<pyscad> 
%<pyscad> module dxfwritelgV(first) {
%<pyscad> 	writedxflgV(first);
%<pyscad> }
%<pyscad> 
%<pyscad> module dxfwritesmbl(first) {
%<pyscad> 	writedxfsmbl(first);
%<pyscad> }
%<pyscad> 
%<pyscad> module dxfwritesmsq(first) {
%<pyscad> 	writedxfsmsq(first);
%<pyscad> }
%<pyscad> 
%<pyscad> module dxfwritesmV(first) {
%<pyscad> 	writedxfsmV(first);
%<pyscad> }
%<pyscad> 
%<pyscad> module svgwriteone(first) {
%<pyscad> 	writesvg(first);
%<pyscad> }
%<pyscad> 
%<pyscad> module writesvglineend(first) {
%<pyscad> 	pwritesvgline();
%<pyscad> }
%<pyscad> 
%<pyscad> module owriteone(first) {
%<pyscad> 	writeln(first);
%<pyscad> }
%<pyscad> 
%<pyscad> module owritetwo(first, second) {
%<pyscad> 	writeln(first, second);
%<pyscad> }
%<pyscad> 
%<pyscad> module owritethree(first, second, third) {
%<pyscad> 	writeln(first, second, third);
%<pyscad> }
%<pyscad> 
%<pyscad> module owritefour(first, second, third, fourth) {
%<pyscad> 	writeln(first, second, third, fourth);
%<pyscad> }
%<pyscad> 
%<pyscad> module owritefive(first, second, third, fourth, fifth) {
%<pyscad> 	writeln(first, second, third, fourth, fifth);
%<pyscad> }
%<pyscad> 
%<pyscad> module owritesix(first, second, third, fourth, fifth, sixth) {
%<pyscad> 	writeln(first, second, third, fourth, fifth, sixth);
%<pyscad> }
%<pyscad> 
%<pyscad> module owriteseven(first, second, third, fourth, fifth, sixth, seventh) {
%<pyscad> 	writeln(first, second, third, fourth, fifth, sixth, seventh);
%<pyscad> }
%<pyscad> 
%<pyscad> module owriteeight(first, second, third, fourth, fifth, sixth, seventh,eighth) {
%<pyscad> 	writeln(first, second, third, fourth, fifth, sixth, seventh,eighth);
%<pyscad> }
%<pyscad> 
%<pyscad> module owritenine(first, second, third, fourth, fifth, sixth, seventh, eighth, ninth) {
%<pyscad> 	writeln(first, second, third, fourth, fifth, sixth, seventh, eighth, ninth);
%<pyscad> }
%<pyscad> 
%<pyscad> module owriteten(first, second, third, fourth, fifth, sixth, seventh, eighth, ninth, tenth) {
%<pyscad> 	writeln(first, second, third, fourth, fifth, sixth, seventh, eighth, ninth, tenth);
%<pyscad> }
%<pyscad> 
%<pyscad> module owriteeleven(first, second, third, fourth, fifth, sixth, seventh, eighth, ninth, tenth, eleventh) {
%<pyscad> 	writeln(first, second, third, fourth, fifth, sixth, seventh, eighth, ninth, tenth, eleventh);
%<pyscad> }
%<pyscad> 
%<pyscad> module owritetwelve(first, second, third, fourth, fifth, sixth, seventh, eighth, ninth, tenth, eleventh, twelfth) {
%<pyscad> 	writeln(first, second, third, fourth, fifth, sixth, seventh, eighth, ninth, tenth, eleventh, twelfth);
%<pyscad> }
%<pyscad> 
%<pyscad> module owritethirteen(first, second, third, fourth, fifth, sixth, seventh, eighth, ninth, tenth, eleventh, twelfth, thirteenth) {
%<pyscad> 	writeln(first, second, third, fourth, fifth, sixth, seventh, eighth, ninth, tenth, eleventh, twelfth, thirteenth);
%<pyscad> }
%<pyscad> 
%    \end{macrocode}
% 
% \DescribeRoutine{dxfwrite}\DescribeRoutine{dxfpreamble}\DescribeRoutine{writesvgline}
%    \begin{macrocode}
%<gcpscad> module dxfwrite(tn,arg) {
%<gcpscad> if (tn == large_ball_tool_no) {
%<gcpscad>     dxfwritelgbl(arg);}
%<gcpscad> if (tn == large_square_tool_no) {
%<gcpscad>     dxfwritelgsq(arg);}
%<gcpscad> if (tn == large_V_tool_no) {
%<gcpscad>     dxfwritelgV(arg);}
%<gcpscad> if (tn == small_ball_tool_no) {
%<gcpscad>     dxfwritesmbl(arg);}
%<gcpscad> if (tn == small_square_tool_no) {
%<gcpscad>     dxfwritesmsq(arg);}
%<gcpscad> if (tn == small_V_tool_no) {
%<gcpscad>     dxfwritesmV(arg);}
%<gcpscad> }
%<gcpscad> 
%<gcpscad> module dxfpreamble(tn) {
%<gcpscad> //    echo(str("dxfpreamble",small_square_tool_no));
%<gcpscad>     dxfwrite(tn,"0");
%<gcpscad>     dxfwrite(tn,"SECTION");
%<gcpscad>     dxfwrite(tn,"2");
%<gcpscad>     dxfwrite(tn,"ENTITIES");
%<gcpscad>     dxfwrite(tn,"0");
%<gcpscad> }
%<gcpscad> 
%<gcpscad> module writesvgline(bx,by,ex,ey) {
%<gcpscad> if (generatesvg == true) {
%<gcpscad>     svgwriteone(str("<path d=",chr(34),"M",bx*3.77953," ",by*3.77953," L",ex*3.77953," ",ey*3.77953," ",chr(34)," stroke=",chr(34),"black",chr(34)," stroke-width=",chr(34),"1",chr(34)," fill=",chr(34),"none",chr(34)," /> "));
%<gcpscad>     }
%<gcpscad> }
%<gcpscad> 
%<gcpscad> module dxfbpl(tn,bx,by) {
%<gcpscad>     dxfwrite(tn,"POLYLINE");
%<gcpscad>     dxfwrite(tn,"8");
%<gcpscad>     dxfwrite(tn,"default");
%<gcpscad>     dxfwrite(tn,"66");
%<gcpscad>     dxfwrite(tn,"1");
%<gcpscad>     dxfwrite(tn,"70");
%<gcpscad>     dxfwrite(tn,"0");
%<gcpscad>     dxfwrite(tn,"0");
%<gcpscad>     dxfwrite(tn,"VERTEX");
%<gcpscad>     dxfwrite(tn,"8");
%<gcpscad>     dxfwrite(tn,"default");
%<gcpscad>     dxfwrite(tn,"70");
%<gcpscad>     dxfwrite(tn,"32");
%<gcpscad>     dxfwrite(tn,"10");
%<gcpscad>     dxfwrite(tn,str(bx));
%<gcpscad>     dxfwrite(tn,"20");
%<gcpscad>     dxfwrite(tn,str(by));
%<gcpscad>     dxfwrite(tn,"0");
%<gcpscad> }
%<gcpscad> 
%<gcpscad> module beginpolyline(bx,by,bz) {
%<gcpscad> if (generatedxf == true) {
%<gcpscad>     dxfwriteone("POLYLINE");
%<gcpscad>     dxfwriteone("8");
%<gcpscad>     dxfwriteone("default");
%<gcpscad>     dxfwriteone("66");
%<gcpscad>     dxfwriteone("1");
%<gcpscad>     dxfwriteone("70");
%<gcpscad>     dxfwriteone("0");
%<gcpscad>     dxfwriteone("0");
%<gcpscad>     dxfwriteone("VERTEX");
%<gcpscad>     dxfwriteone("8");
%<gcpscad>     dxfwriteone("default");
%<gcpscad>     dxfwriteone("70");
%<gcpscad>     dxfwriteone("32");
%<gcpscad>     dxfwriteone("10");
%<gcpscad>     dxfwriteone(str(bx));
%<gcpscad>     dxfwriteone("20");
%<gcpscad>     dxfwriteone(str(by));
%<gcpscad>     dxfwriteone("0");
%<gcpscad>     dxfbpl(current_tool(),bx,by);}
%<gcpscad> }
%<gcpscad> 
%<gcpscad> module dxfapl(tn,bx,by) {
%<gcpscad>     dxfwrite(tn,"VERTEX");
%<gcpscad>     dxfwrite(tn,"8");
%<gcpscad>     dxfwrite(tn,"default");
%<gcpscad>     dxfwrite(tn,"70");
%<gcpscad>     dxfwrite(tn,"32");
%<gcpscad>     dxfwrite(tn,"10");
%<gcpscad>     dxfwrite(tn,str(bx));
%<gcpscad>     dxfwrite(tn,"20");
%<gcpscad>     dxfwrite(tn,str(by));
%<gcpscad>     dxfwrite(tn,"0");
%<gcpscad> }
%<gcpscad> 
%<gcpscad> module addpolyline(bx,by,bz) {
%<gcpscad> if (generatedxf == true) {
%<gcpscad>     dxfwriteone("VERTEX");
%<gcpscad>     dxfwriteone("8");
%<gcpscad>     dxfwriteone("default");
%<gcpscad>     dxfwriteone("70");
%<gcpscad>     dxfwriteone("32");
%<gcpscad>     dxfwriteone("10");
%<gcpscad>     dxfwriteone(str(bx));
%<gcpscad>     dxfwriteone("20");
%<gcpscad>     dxfwriteone(str(by));
%<gcpscad>     dxfwriteone("0");
%<gcpscad>     dxfapl(current_tool(),bx,by);
%<gcpscad>     }
%<gcpscad> }
%<gcpscad> 
%<gcpscad> module dxfcpl(tn) {
%<gcpscad>     dxfwrite(tn,"SEQEND");
%<gcpscad>     dxfwrite(tn,"0");
%<gcpscad> }
%<gcpscad> 
%<gcpscad> module closepolyline() {
%<gcpscad> if (generatedxf == true) {
%<gcpscad>     dxfwriteone("SEQEND");
%<gcpscad>     dxfwriteone("0");
%<gcpscad>     dxfcpl(current_tool());
%<gcpscad>     }
%<gcpscad> }
%<gcpscad> 
%<gcpscad> module writecomment(comment) {
%<gcpscad> if (generategcode == true) {
%<gcpscad> 	owritecomment(comment);
%<gcpscad> }
%<gcpscad> }
%<gcpscad> 
%    \end{macrocode}
% 
% \DescribeRoutine{pclosegcodefile}\DescribeRoutine{pclosesvgfile}\DescribeRoutine{pclosedxffile}
% At the end of the project it will be necessary to close each file. 
% In some instances it will be necessary to write additional information,
% depending on the file format.
% 
%    \begin{macrocode}
%<gcpy>def pclosegcodefile():
%<gcpy>    f.close()
%<gcpy>
%<gcpy>def pclosesvgfile():
%<gcpy>    svg.close()
%<gcpy>
%<gcpy>def pclosedxffile():
%<gcpy>    dxf.close()
%<gcpy>
%<gcpy>def pclosedxflgblfile():
%<gcpy>    dxflgbl.close()
%<gcpy>
%<gcpy>def pclosedxflgsqfile():
%<gcpy>    dxflgsq.close()
%<gcpy>
%<gcpy>def pclosedxflgVfile():
%<gcpy>    dxflgV.close()
%<gcpy>
%<gcpy>def pclosedxfsmblfile():
%<gcpy>    dxfsmbl.close()
%<gcpy>
%<gcpy>def pclosedxfsmsqfile():
%<gcpy>    dxfsmsq.close()
%<gcpy>
%<gcpy>def pclosedxfsmVfile():
%<gcpy>    dxfsmV.close()
%<gcpy>
%    \end{macrocode}
% 
% \DescribeRoutine{oclosegcodefile}\DescribeRoutine{oclosedxffile}\DescribeRoutine{oclosedxflgblfile}
%    \begin{macrocode}
%<pyscad> module oclosegcodefile() {
%<pyscad> 	pclosegcodefile();
%<pyscad> }
%<pyscad> 
%<pyscad> module oclosedxffile() {
%<pyscad> 	pclosedxffile();
%<pyscad> }
%<pyscad> 
%<pyscad> module oclosedxflgblfile() {
%<pyscad> 	pclosedxflgblfile();
%<pyscad> }
%<pyscad> 
%<pyscad> module oclosedxflgsqfile() {
%<pyscad> 	pclosedxflgsqfile();
%<pyscad> }
%<pyscad> 
%<pyscad> module oclosedxflgVfile() {
%<pyscad> 	pclosedxflgVfile();
%<pyscad> }
%<pyscad> 
%<pyscad> module oclosedxfsmblfile() {
%<pyscad> 	pclosedxfsmblfile();
%<pyscad> }
%<pyscad> 
%<pyscad> module oclosedxfsmsqfile() {
%<pyscad> 	pclosedxfsmsqfile();
%<pyscad> }
%<pyscad> 
%<pyscad> module oclosedxfsmVfile() {
%<pyscad> 	pclosedxfsmVfile();
%<pyscad> }
%<pyscad> 
%<pyscad> module oclosesvgfile() {
%<pyscad> 	pclosesvgfile();
%<pyscad> }
%<pyscad> 
%    \end{macrocode}
% 
% \DescribeRoutine{closegcodefile}\DescribeRoutine{dxfpostamble}\DescribeRoutine{closedxffile}
%    \begin{macrocode}
%<gcpscad> module closegcodefile() {
%<gcpscad> if (generategcode == true) {
%<gcpscad>     owriteone("M05");
%<gcpscad>     owriteone("M02");
%<gcpscad> 	oclosegcodefile();
%<gcpscad> }
%<gcpscad> }
%<gcpscad> 
%<gcpscad> module dxfpostamble(arg) {
%<gcpscad>     dxfwrite(arg,"ENDSEC");
%<gcpscad>     dxfwrite(arg,"0");
%<gcpscad>     dxfwrite(arg,"EOF");
%<gcpscad> }
%<gcpscad> 
%<gcpscad> module closedxffile() {
%<gcpscad> if (generatedxf == true) {
%<gcpscad>     dxfwriteone("ENDSEC");
%<gcpscad>     dxfwriteone("0");
%<gcpscad>     dxfwriteone("EOF");
%<gcpscad> 	oclosedxffile();
%<gcpscad>     echo("CLOSING");
%<gcpscad> if (large_ball_tool_no >  0) {	dxfpostamble(large_ball_tool_no);
%<gcpscad>     oclosedxflgblfile();
%<gcpscad> } 
%<gcpscad> if (large_square_tool_no >  0) {	dxfpostamble(large_square_tool_no);
%<gcpscad>     oclosedxflgsqfile();
%<gcpscad> } 
%<gcpscad> if (large_V_tool_no >  0) {	dxfpostamble(large_V_tool_no);
%<gcpscad>     oclosedxflgVfile();
%<gcpscad> } 
%<gcpscad> if (small_ball_tool_no >  0) {	dxfpostamble(small_ball_tool_no);
%<gcpscad>     oclosedxfsmblfile();
%<gcpscad> } 
%<gcpscad> if (small_square_tool_no >  0) {	dxfpostamble(small_square_tool_no);
%<gcpscad>     oclosedxfsmsqfile();
%<gcpscad> } 
%<gcpscad> if (small_V_tool_no >  0) {	dxfpostamble(small_V_tool_no);
%<gcpscad>     oclosedxfsmVfile();
%<gcpscad> } 
%<gcpscad>     }
%<gcpscad> }
%<gcpscad> 
%<gcpscad> module closesvgfile() {
%<gcpscad> if (generatesvg == true) {
%<gcpscad>     svgwriteone("</svg> ");
%<gcpscad> 	oclosesvgfile();
%<gcpscad>     echo("CLOSING SVG");
%<gcpscad>     }
%<gcpscad> }
%<gcpscad> 
%    \end{macrocode}
% 
% \subsection{Movement and Cutting}
% 
% \DescribeRoutine{otm}\DescribeRoutine{ocut}\DescribeRoutine{orapid}
% With all the scaffolding in place, it is possible to model tool movement
% and cutting and to write out files which represent the desired machine motions.
% 
%    \begin{macrocode}
%<gcpscad> module otm(ex, ey, ez, r,g,b) {
%<gcpscad> color([r,g,b]) hull(){
%<gcpscad>     translate([xpos(), ypos(), zpos()]){
%<gcpscad>       select_tool(current_tool());
%<gcpscad>     }
%<gcpscad>     translate([ex, ey, ez]){
%<gcpscad>       select_tool(current_tool());
%<gcpscad>     }
%<gcpscad>   }
%<gcpscad> oset(ex, ey, ez);
%<gcpscad> }
%<gcpscad> 
%<gcpscad> module ocut(ex, ey, ez) {
%<gcpscad> //color([0.2,1,0.2]) hull(){
%<gcpscad> otm(ex, ey, ez, 0.2,1,0.2);
%<gcpscad> }
%<gcpscad> 
%<gcpscad> module orapid(ex, ey, ez) {
%<gcpscad> //color([0.93,0,0]) hull(){
%<gcpscad> otm(ex, ey, ez, 0.93,0,0);
%<gcpscad> }
%<gcpscad> 
%<gcpscad> module rapidbx(bx, by, bz, ex, ey, ez) {
%<gcpscad> //	writeln("G0 X",bx," Y", by, "Z", bz);
%<gcpscad> if (generategcode == true) {
%<gcpscad> 	writecomment("rapid");
%<gcpscad> 	owritesix("G0 X",str(ex)," Y", str(ey), " Z", str(ez));
%<gcpscad> }
%<gcpscad>     orapid(ex, ey, ez);
%<gcpscad> }
%<gcpscad> 
%<gcpscad> module rapid(ex, ey, ez) {
%<gcpscad> //	writeln("G0 X",bx," Y", by, "Z", bz);
%<gcpscad> if (generategcode == true) {
%<gcpscad> 	writecomment("rapid");
%<gcpscad> 	owritesix("G0 X",str(ex)," Y", str(ey), " Z", str(ez));
%<gcpscad> }
%<gcpscad>     orapid(ex, ey, ez);
%<gcpscad> }
%<gcpscad> 
%<gcpscad> module movetosafez() {
%<gcpscad> //this should be move to retract height
%<gcpscad> if (generategcode == true) {
%<gcpscad> 	writecomment("Move to safe Z to avoid workholding");
%<gcpscad>     owriteone("G53G0Z-5.000");
%<gcpscad> }
%<gcpscad>     orapid(getxpos(), getypos(), retractheight+55);
%<gcpscad> }
%<gcpscad> 
%<gcpscad> module begintoolpath(bx,by,bz) {
%<gcpscad> if (generategcode == true) {
%<gcpscad> 	writecomment("PREPOSITION FOR RAPID PLUNGE");
%<gcpscad>     owritefour("G0X", str(bx), "Y",str(by));
%<gcpscad>     owritetwo("Z", str(bz));
%<gcpscad>     }
%<gcpscad>     orapid(bx,by,bz);
%<gcpscad> }
%<gcpscad> 
%<gcpscad> module movetosafeheight() {
%<gcpscad> //this should be move to machine position
%<gcpscad> if (generategcode == true) {
%<gcpscad> //	writecomment("PREPOSITION FOR RAPID PLUNGE");Z25.650
%<gcpscad> //G1Z24.663F381.0 ,"F",str(plunge)
%<gcpscad> if (zeroheight == "Top") {
%<gcpscad>     owritetwo("Z",str(retractheight));
%<gcpscad> }
%<gcpscad> }
%<gcpscad>     orapid(getxpos(), getypos(), retractheight+55);
%<gcpscad> }
%<gcpscad> 
%<gcpscad> module cutoneaxis_setfeed(axis,depth,feed) {
%<gcpscad> if (generategcode == true) {
%<gcpscad> //	writecomment("PREPOSITION FOR RAPID PLUNGE");Z25.650
%<gcpscad> //G1Z24.663F381.0 ,"F",str(plunge) G1Z7.612F381.0
%<gcpscad> if (zeroheight == "Top") {
%<gcpscad>     owritefive("G1",axis,str(depth),"F",str(feed));
%<gcpscad> }
%<gcpscad> }
%<gcpscad> if (axis == "X") {setxpos(depth);}
%<gcpscad> if (axis == "Y") {setypos(depth);}
%<gcpscad> if (axis == "Z") {setzpos(depth);}
%<gcpscad> }
%<gcpscad> 
%<gcpscad> module cut(ex, ey, ez) {
%<gcpscad> //	writeln("G0 X",bx," Y", by, "Z", bz);
%<gcpscad> if (generategcode == true) {
%<gcpscad> //	writecomment("rapid");
%<gcpscad> 	owritesix("G1 X",str(ex)," Y", str(ey), " Z", str(ez));
%<gcpscad> }
%<gcpscad> if (generatesvg == true) {
%<gcpscad> //	owritesix("G1 X",str(ex)," Y", str(ey), " Z", str(ez));
%<gcpscad> //    orapid(getxpos(), getypos(), retractheight+5);
%<gcpscad>     writesvgline(getxpos(),getypos(),ex,ey);
%<gcpscad> }
%<gcpscad> ocut(ex, ey, ez);
%<gcpscad> }
%<gcpscad> 
%<gcpscad> module cutwithfeed(ex, ey, ez, feed) {
%<gcpscad> //	writeln("G0 X",bx," Y", by, "Z", bz);
%<gcpscad> if (generategcode == true) {
%<gcpscad> //	writecomment("rapid");
%<gcpscad> 	owriteeight("G1 X",str(ex)," Y", str(ey), " Z", str(ez),"F",str(feed));
%<gcpscad> }
%<gcpscad> ocut(ex, ey, ez);
%<gcpscad> }
%<gcpscad> 
%<gcpscad> module endtoolpath() {
%<gcpscad> if (generategcode == true) {
%<gcpscad> //Z31.750
%<gcpscad> //	owriteone("G53G0Z-5.000");
%<gcpscad>     owritetwo("Z",str(retractheight));
%<gcpscad> }
%<gcpscad>     orapid(getxpos(),getypos(),retractheight);
%<gcpscad> }
%    \end{macrocode}
% 
% \section{gcodepreview\_template.scad}
% 
%    \begin{macrocode}
%<gcptmpl> //!OpenSCAD
%<gcptmpl> 
%<gcptmpl> use <gcodepreview.py>;
%<gcptmpl> use <pygcodepreview.scad>;
%<gcptmpl> include <gcodepreview.scad>;
%<gcptmpl> 
%<gcptmpl> $fa = 2;
%<gcptmpl> $fs = 0.125;
%<gcptmpl> 
%<gcptmpl> /* [Export] */
%<gcptmpl> Base_filename = "export"; 
%<gcptmpl> 
%<gcptmpl> /* [Export] */
%<gcptmpl> generatedxf = true; 
%<gcptmpl> 
%<gcptmpl> /* [Export] */
%<gcptmpl> generategcode = true; 
%<gcptmpl> 
%<gcptmpl> /* [Export] */
%<gcptmpl> generatesvg = false; 
%<gcptmpl> 
%<gcptmpl> /* [CAM] */
%<gcptmpl> toolradius = 1.5875;
%<gcptmpl> /* [CAM] */
%<gcptmpl> large_ball_tool_no = 0; // [0:0,111:111,101:101,202:202]
%<gcptmpl> 
%<gcptmpl> /* [CAM] */
%<gcptmpl> large_square_tool_no = 0; // [0:0,112:112,102:102,201:201]
%<gcptmpl> 
%<gcptmpl> /* [CAM] */
%<gcptmpl> large_V_tool_no = 0; // [0:0,301:301,690:690]
%<gcptmpl> 
%<gcptmpl> /* [CAM] */
%<gcptmpl> small_ball_tool_no = 0; // [0:0,121:121,111:111,101:101]
%<gcptmpl> 
%<gcptmpl> /* [CAM] */
%<gcptmpl> small_square_tool_no = 102; // [0:0,122:122,112:112,102:102]
%<gcptmpl> 
%<gcptmpl> /* [CAM] */
%<gcptmpl> small_V_tool_no = 0; // [0:0,390:390,301:301]
%<gcptmpl> 
%<gcptmpl> /* [Feeds and Speeds] */
%<gcptmpl> plunge = 100;
%<gcptmpl> /* [Feeds and Speeds] */
%<gcptmpl> feed = 400;
%<gcptmpl> /* [Feeds and Speeds] */
%<gcptmpl> speed = 16000;
%<gcptmpl> /* [Feeds and Speeds] */
%<gcptmpl> square_ratio = 1.0; // [0.25:2]
%<gcptmpl> /* [Feeds and Speeds] */
%<gcptmpl> small_V_ratio = 0.75; // [0.25:2]
%<gcptmpl> /* [Feeds and Speeds] */
%<gcptmpl> large_V_ratio = 0.875; // [0.25:2]
%<gcptmpl> 
%<gcptmpl> /* [Stock] */
%<gcptmpl> stocklength = 219;
%<gcptmpl> /* [Stock] */
%<gcptmpl> stockwidth = 150;
%<gcptmpl> /* [Stock] */
%<gcptmpl> stockthickness = 8.35;
%<gcptmpl> /* [Stock] */
%<gcptmpl> zeroheight = "Top"; // [Top, Bottom]
%<gcptmpl> /* [Stock] */
%<gcptmpl> stockorigin = "Center"; // [Lower-Left, Center-Left, Top-Left, Center]
%<gcptmpl> /* [Stock] */
%<gcptmpl> retractheight = 9;
%<gcptmpl> 
%<gcptmpl> filename_gcode = str(Base_filename, ".nc");
%<gcptmpl> filename_dxf = str(Base_filename);
%<gcptmpl> filename_svg = str(Base_filename, ".svg");
%<gcptmpl> 
%<gcptmpl> opengcodefile(filename_gcode);
%<gcptmpl> opendxffile(filename_dxf);
%<gcptmpl> 
%<gcptmpl> difference() {
%<gcptmpl> setupstock(stocklength, stockwidth, stockthickness, zeroheight, stockorigin);
%<gcptmpl> 
%<gcptmpl> movetosafez();
%<gcptmpl> 
%<gcptmpl> toolchange(small_square_tool_no,speed * square_ratio);
%<gcptmpl> 
%<gcptmpl> begintoolpath(0,0,0.25);
%<gcptmpl> beginpolyline(0,0,0.25);
%<gcptmpl> 
%<gcptmpl> cutoneaxis_setfeed("Z",-1,plunge*square_ratio);
%<gcptmpl> 
%<gcptmpl> cutwithfeed(stocklength/2,stockwidth/2,-stockthickness,feed);
%<gcptmpl> addpolyline(stocklength/2,stockwidth/2,-stockthickness);
%<gcptmpl> 
%<gcptmpl> endtoolpath();
%<gcptmpl> closepolyline();
%<gcptmpl> }
%<gcptmpl> 
%<gcptmpl> closegcodefile();
%<gcptmpl> closedxffile();
%    \end{macrocode}
% 
% \bibliographystyle{alpha}
%
% \begin{thebibliography}{RS274}
%
% \bibitem[RS274]{KRAMER00}
% Thomas R. Kramer, Frederick M. Proctor, Elena R. Messina.\\
% \url{https://tsapps.nist.gov/publication/get_pdf.cfm?pub_id=823374}
%
% \end{thebibliography}
%
%
% \Finale
%
% \DoNotIndex{\\\\,\\~,\n,\Users,\RapCAD}
% 
% \PrintIndex
%
\endinput
