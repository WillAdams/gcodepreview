% \iffalse meta-comment
%
% gcodepreview.dtx
% Author: William F. Adams (willadams at aol dot com)
% Copyright 2021--24 William F. Adams
%
% This work may be distributed and/or modified under the
% conditions of the GNU LESSER GENERAL PUBLIC LICENSE
% Version 2.1, February 1999
%
% This work consists of the files listed in the README file.
%
% 
% 
%<*driver>
\documentclass[twoside]{ltxdoc}
\usepackage{docmfp}
\usepackage{url}
\usepackage[draft=false,
            plainpages=false,
            pdfpagelabels,
            bookmarksnumbered,
            hyperindex=false
           ]{hyperref}
\makeatletter
  \@mparswitchfalse
\makeatother
\renewcommand{\MakeUppercase}[1]{#1}
\pagestyle{headings}
\EnableCrossrefs
\CodelineIndex
%%% \OnlyDescription
\setcounter{StandardModuleDepth}{1}
\begin{document}
  \DocInput{gcodepreview.dtx}
\end{document}
%</driver>
%
% \fi
%
% \DoNotIndex{\bullet}
%
%
% \changes{v0.2}{2024/04/12}{Initial conversion to DTX}
% \def\dtxfile{gcodepreview.dtx}
% \def\fileversion{v0.2} \def\filedate{2024/04/12}
%
% \title{The gcodepreview OpenSCAD library\thanks{This
%        file (\texttt{\dtxfile}) has version number \fileversion, last revised
%        \filedate.}}
%
% \author{%
% Author: William F. Adams\\
% \texttt{willadams at aol dot com}
% }
% \date{\filedate}
% \maketitle
% \begin{abstract}
%    The gcodepreview library allows using OpenSCAD to move a tool in lines 
%and output dxf and G-code files so as to work as a CAD/CAM program for CNC.
% \end{abstract}
% \tableofcontents
%
%
% \section{readme.md}
%
%    \begin{macrocode}
%<rdm> # gcodepreview
%<rdm> 
%<rdm> OpenSCAD library for moving a tool in lines and arcs so as to model how a part would be cut 
%<rdm> using G-Code, so as to allow OpenSCAD to function as a compleat CAD/CAM solution for 
%<rdm> subtractive CNC (mills and routers).
%<rdm> 
%<rdm> ![OpenSCAD Cut Joinery Module](https://raw.githubusercontent.com/WillAdams/gcodepreview/main/openscad_cutjoinery.png?raw=true)
%<rdm> 
%<rdm> Updated to make use of Python in OpenSCAD:
%<rdm> 
%<rdm> http://www.guenther-sohler.net/openscad/
%<rdm> 
%<rdm> (previous versions had used RapCAD)
%<rdm> 
%<rdm> A BlockSCAD file for the main modules is available at:
%<rdm> 
%<rdm> https://www.blockscad3d.com/community/projects/1244473
%<rdm> 
%<rdm> The project is discussed at:
%<rdm> 
%<rdm> https://forum.makerforums.info/t/g-code-preview-using-openscad-rapcad/85729 
%<rdm> 
%<rdm> and
%<rdm> 
%<rdm> https://forum.makerforums.info/t/openscad-and-python-looking-to-finally-be-resolved/88171
%<rdm> 
%<rdm> and
%<rdm> 
%<rdm> https://willadams.gitbook.io/design-into-3d/programming
%<rdm> 
%<rdm> Usage is:
%<rdm> 
%<rdm> Place the file in C:\Users\\\~\Documents\OpenSCAD\libraries (C:\Users\\\~\Documents\RapCAD\libraries is 
%<rdm> deprecated since RapCAD is not longer needed since Python is now used for writing out files)
%<rdm> 
%<rdm> (While it was updated for use w/ RapCAD, so as to take advantage of the writeln command, 
%<rdm> it was possible to write that in Python)
%<rdm> 
%<rdm>     use <gcodepreview.py>;
%<rdm>     use <pygcodepreview.scad>;
%<rdm>     include <gcodepreview.scad>;
%<rdm> 
%<rdm> Note that it is necessary to use the first two files (this allows loading 
%<rdm> the Python commands and then wrapping them in OpenSCAD commands) and then 
%<rdm> include the last file (which allows using OpenSCAD variables to selectively 
%<rdm> implement the Python commands via their being wrapped in OpenSCAD modules)
%<rdm> 
%<rdm> and define variables which match the project and then use commands such as:
%<rdm> 
%<rdm>     opengcodefile(Gcode_filename);
%<rdm>     opendxffile(DXF_filename);
%<rdm>     
%<rdm>     difference() {
%<rdm>         setupstock(stocklength, stockwidth, stockthickness, zeroheight, stockorigin);
%<rdm>     
%<rdm>     movetosafez();
%<rdm>     
%<rdm>     toolchange(squaretoolno,speed * square_ratio);
%<rdm>     
%<rdm>     begintoolpath(0,0,0.25);
%<rdm>     beginpolyline(0,0,0.25);
%<rdm> 
%<rdm>     cutoneaxis_setfeed("Z",-1,plunge*square_ratio);
%<rdm>     addpolyline(stocklength/2,stockwidth/2,-stockthickness);
%<rdm>     
%<rdm>     cutwithfeed(stocklength/2,stockwidth/2,-stockthickness,feed);
%<rdm>     
%<rdm>     endtoolpath();
%<rdm>     endpolyline();
%<rdm>     
%<rdm>     }
%<rdm>     
%<rdm>     closegcodefile();
%<rdm>     closedxffile();
%<rdm> 
%<rdm> Tool numbers match those of tooling sold by Carbide 3D (ob. discl., I work for them). 
%<rdm> Comments are included in the G-code to match those expected by CutViewer.
%<rdm> 
%<rdm> A complete example file is: gcodepreview_template.scad another is 
%<rdm> openscad_gcodepreview_cutjoinery.tres.scad which is made from an 
%<rdm> OpenSCAD Graph Editor file:
%<rdm> 
%<rdm> ![OpenSCAD Graph Editor Cut Joinery File](https://raw.githubusercontent.com/WillAdams/gcodepreview/main/OSGE_cutjoinery.png?raw=true)
%<rdm> 
%<rdm> Version 0.1 supports setting up stock, origin, rapid positioning, making cuts, 
%<rdm> and writing out matching G-code, and creating a DXF with polylines.
%<rdm> 
%<rdm> Added features since initial upload:
%<rdm> 
%<rdm>  - endpolyline(); --- this command allows ending one polyline so as to allow multiple lines in a DXF
%<rdm>  - separate dxf files are written out for each tool where tool is ball/square/V and small/large (10/31)
%<rdm> 
%<rdm> Not quite working feature:
%<rdm> 
%<rdm>  - exporting SVGs --- these are written out upside down due to coordinate differences between OpenSCAD/DXFs and SVGs
%<rdm> 
%<rdm> Possible future improvements:
%<rdm> 
%<rdm>  - G-code: support for G2/G3 arcs
%<rdm>  - DXF support for curves and the 3rd dimension
%<rdm>  - G-code: import external tool libraries and feeds and speeds from JSON or CSV files --- note that it is up to the user to implement Depth per Pass so as to not take a single full-depth pass
%<rdm>  - support for additional tooling shapes such as dovetail tools, or roundover tooling
%<rdm>  - general coding improvements --- current coding style is quite prosaic
%<rdm>  - documentation --- this will probably be at: https://willadams.gitbook.io/design-into-3d/programming
%<rdm>  - possibly re-writing as a literate program
%<rdm>  - generalized modules for cutting out various shapes/geometries --- a current one is to cut a rectangular area as vertical passes (the horizontal version will be developed presently)
%
%    \end{macrocode}
%
% \section{gcodepreview}
%
% As noted above, this library works by using Python code as a back-end so as to persistently
% store and access variables, and to write out files. Doing so requires a total of three files:
%
%  - A Python file: gcodepreview.py (gcpy)
%  - An OpenSCAD file: gcodepreview.scad (gcpscad)
%  - An OpenSCAD file which connects the other two files: pygcodepreview.scad (pyscad)
%
%
% \bibliographystyle{alpha}
%
% \begin{thebibliography}{RS274}
%
% \bibitem[RS274]{KRAMER00}
% Thomas R. Kramer, Frederick M. Proctor, Elena R. Messina.\\
% \url{https://tsapps.nist.gov/publication/get_pdf.cfm?pub_id=823374}
%
% \end{thebibliography}
%
%
% \Finale
% \PrintIndex
%
\endinput
